\documentclass[a4paper, 10pt]{article}
\usepackage[a4paper,left=3cm,right=2cm,top=2.5cm,bottom=2.5cm]{geometry}
\usepackage[utf8]{inputenc} % Change according your file encoding
\usepackage{graphicx}
%\usepackage[demo]{graphicx}
\usepackage{url}

\usepackage{float}
\usepackage{amsmath}
\usepackage{xcolor}
\usepackage{todonotes}

\usepackage{listings}

\definecolor{backcolour}{rgb}{0.95,0.95,0.92}

\lstdefinestyle{mystyle}{
    backgroundcolor=\color{backcolour},  
    breakatwhitespace=false,         
    basicstyle=\scriptsize,
    breaklines=true,                 
    captionpos=b,                    
    keepspaces=true,                 
    showspaces=false,                
    showstringspaces=false,
    showtabs=false,                  
    tabsize=2,
    frame=single
}



\lstset{style=mystyle}

%opening
\title{Algorithmic Methods for Mathematical Models\\Course Project}
\author{Ignacio Encinas Rubio, Adrián Jimenez González}
\date{\normalsize\today{}}

\begin{document}

\maketitle

\section{Problem statement}
\textit{The formal problem statement}

\section{Integer Linear Programming Model}
\textit{The integer linear programming model, with a definition and a short description of the variables, the objective function and the constraints. Do not include OPL code in the document, but rather their mathematical formulation.}

Mirar los enunciados de las prácticas, en concreto Lab3

\section{Meta-heuristics}
\textit{For the meta-heuristics, the pseudo-code of your constructive, local search and GRASP algorithms, including equations for describing the greedy cost function(s) and the RCL}

\subsection{Constructive}

\subsection{Local search}

\subsection{GRASP}

\section{Parameter tuning}

\section{Results}

\section{Reproducing the results}
\todo[inline]{maybe setup a script to reproduce results or something...}

\begin{itemize}
    \item OPL source code
    \item Programs of the meta-heuristics
    \item Instance generator
    \item Instructions how to use every of them and how to reproduce results
\end{itemize}



\end{document}

% !TEX encoding = UTF-8 Unicode 
\documentclass[9pt, aspectratio=169, xcolor=table]{beamer}
\setbeamercovered{transparent=10}
\usetheme[
%  showheader,
  colorblocks,
%  noframetitlerule,
]{Verona}
\definecolor{deepbullblue}{rgb}{0,0.298, 0.427}
\definecolor{sunbeltyellow}{rgb}{1,0.792, 0.024}
\definecolor{redbullred}{rgb}{0.8,0.122, 0.294}
\definecolor{mGreen}{rgb}{0,0.6,0}
\definecolor{mGray}{rgb}{0.5,0.5,0.5}
\definecolor{grissuave}{rgb}{0.25,0.25,0.25}
\definecolor{mPurple}{rgb}{0.58,0,0.82}
\definecolor{backgroundColour}{rgb}{0.95,0.95,0.92}

\definecolor{alizarin}{rgb}{0.82, 0.1, 0.26}
\definecolor{ao}{rgb}{0.0, 0.5, 0.0}

\usepackage{minted}
\usepackage{svg}
\usepackage{listings}

\usepackage[spanish,es-tabla]{babel}

\lstdefinestyle{CStyle}{
    backgroundcolor=\color{backgroundColour},   
    commentstyle=\color{mGreen},
    keywordstyle=\color{magenta},
    numbers=none,
    %numberstyle=\tiny\color{mGray},
    %numbers=left,                    
    %numbersep=5pt,                  
    stringstyle=\color{mPurple},
    basicstyle=\ttfamily,
    breakatwhitespace=false,         
    breaklines=true,                 
    captionpos=b,                    
    keepspaces=true,                 
    showspaces=false,
    columns=fixed,
    showstringspaces=false,
    showtabs=false,                  
    tabsize=2,
    language=C
}

\usepackage[T1]{fontenc}
\usepackage[utf8]{inputenc}
\usepackage{listings}
\usepackage{datetime}
\usepackage{lipsum}
\usepackage{todonotes}
\usepackage[export]{adjustbox}
%\usepackage[spanish,onelanguage,boxed, boxruled, algoruled]{algorithm2e}
%\usepackage[spanish,onelanguage,boxed, algoruled]{algorithm2e}
%\usepackage{algcompatible}
\usepackage{algorithm}
\usepackage[noend]{algpseudocode}
%%%%%%%%%%%%%%%%%%%%%%%%%%%%%%%
% Mac上使用如下命令声明隶书字体,windows也有相关方式,大家可自行修改
%\providecommand{\lishu}{\CJKfamily{zhli}}
%%%%%%%%%%%%%%%%%%%%%%%%%%%%%%%
\usepackage{tikz}
\usetikzlibrary{fadings}
%
%\setbeamertemplate{sections/subsections in toc}[ball]
%\usepackage{xeCJK}
\usepackage{caption}
\usepackage{subcaption}
\usefonttheme{professionalfonts}
\def\mathfamilydefault{\rmdefault}
\usepackage{amsmath}
\usepackage{multirow}
\usepackage{booktabs}
\usepackage{multicol}
\usepackage{bm}

\def\testclr#1#{\@testclr{#1}}
%\def\@testclr#1#2{\scalebox{0.7}{\colorbox#1{#2}{\phantom{X}}}}
\def\@testclr#1#2#3{{\colorbox#1{#2}{#3}}}

\setbeamertemplate{section in toc}{\hspace*{1em}\inserttocsectionnumber.~\inserttocsection\par}
\setbeamertemplate{subsection in toc}{\hspace*{2em}\inserttocsectionnumber.\inserttocsubsectionnumber.~\inserttocsubsection\par}
\setbeamerfont{subsection in toc}{size=\small}
%\AtBeginSection[]{%
	%\begin{frame}%
		%\frametitle{Contenidos}%
		%%\textbf{\tableofcontents[currentsection]} %
		%\begin{columns}[t]
		    %\begin{column}{.5\textwidth}
			%\tableofcontents[sections={1-3}, currentsection]
		    %\end{column}
		    %\begin{column}{.5\textwidth}
			%\tableofcontents[sections={4-7}, currentsection]
		    %\end{column}
		%\end{columns}
	%\end{frame}%
%}

%\AtBeginLecture{%
%\begin{frame}[plain]
%\begin{center}
	%\begin{tcolorbox}[blanker,borderline horizontal={3pt}{20pt}{structure.fg}]
	  %\centering\color{structure.fg}\bfseries \lecturename~\thelecture.\quad \insertlecture\\
	%\end{tcolorbox}
%\end{center}
%\end{frame}


%\AtBeginLecture{
	%\ifnum\framenumber=0
	%\else
	    %\begin{frame}%
		    %\frametitle{Contenidos}%
		    %%\textbf{\tableofcontents[currentsection]} %
		    %\begin{columns}[t]
			%\begin{column}{.5\textwidth}
			    %\tableofcontents[sections={1-3}, currentsection]
			%\end{column}
			%\begin{column}{.5\textwidth}
			    %\tableofcontents[sections={4-7}, currentsection]
			%\end{column}
		    %\end{columns}
	    %\end{frame}%
	%\fi
%}
%\AtBeginSection[]{%
	%\begin{frame}%
		%\frametitle{ }%
		%\textbf{\tableofcontents[currentsection]} %
	%\end{frame}%
%}

%\AtBeginSection{
	    %\begin{frame}%
		    %\frametitle{Contenidos}%
		    %%\textbf{\tableofcontents[currentsection]} %
		    %\begin{columns}[t]
			%\begin{column}{.5\textwidth}
			    %\tableofcontents[sections={1-2}, currentsection]
			%\end{column}
			%\begin{column}{.5\textwidth}
			    %\tableofcontents[sections={3-7}, currentsection]
			%\end{column}
		    %\end{columns}
	    %\end{frame}%
%}

%\AtBeginSubsection[]{%
%	\begin{frame}%
%		\frametitle{ }%
%		\textbf{\tableofcontents[currentsection, currentsubsection]} %
%	\end{frame}%
%}

\title{AMMM - Course project}
\subtitle{Master in Research and Innovation in Informatics}
\author[Ignacio Encinas Rubio, Adrián Jiménez González]{Ignacio Encinas Rubio, Adrián Jiménez Gonzalez}
\mail{\{ignacio.encinas,adrian.jimenez.g\}@estudiantat.upc.edu}
\institute[]{Polytechnic University of Catalonia}
\date{\today}
\titlegraphic[width=3.5cm]{logo_upc.png}{}

\usepackage{tcolorbox}
\usepackage{fontawesome}

\begin{document}


\maketitle

%%% define code
\defverbatim[colored]\lstI{
	\begin{lstlisting}[language=C++,basicstyle=\ttfamily,keywordstyle=\color{red}]
	int main() {
	// Define variables at the beginning
	// of the block, as in C:
	CStash intStash, stringStash;
	int i;
	char* cp;
	ifstream in;
	string line;
	[...]
	\end{lstlisting}
}
%%%%%%%%%%%%%%%%%%%%%%%%%%%%%%%%
% ----------- FRAME ------------
%%%%%%%%%%%%%%%%%%%%%%%%%%%%%%%%
\begin{frame}%
	\frametitle{Contents}%
	%\textbf{\tableofcontents[currentsection]} %
	\begin{columns}[t]
	    \begin{column}{.5\textwidth}
		\tableofcontents[sections={1-2}]
	    \end{column}
	    \begin{column}{.5\textwidth}
		\tableofcontents[sections={3-5}]
	    \end{column}
	\end{columns}
\end{frame}%

\section{Problem Statement}
\begin{frame}{\secname}
    \begin{tcolorbox}[colback=gray!30, colframe=Veronablue, arc=0pt, outer arc=0pt, title = \textbf{Main requirements}]
    \begin{enumerate}
	\item Each contestant will play exactly once against each of the other contestants.
	\item Each round will consist of $\frac{n-1}{2}$ matches.
	\item Players will play 50\% of their games as white, 50\% will be played as black.
    \end{enumerate}
    \end{tcolorbox}

    \begin{tcolorbox}[colback=gray!30, colframe=Veronablue, arc=0pt, outer arc=0pt, title = \textbf{Subtle requirements}]
    \begin{itemize}
	\item A player can only play up to 1 game per round
	\item A player can't play against himself
    \end{itemize}
    \end{tcolorbox}
    
\end{frame}

\subsection{Inputs \& Outputs}
\begin{frame}{\secname: \subsecname}
    \begin{tcolorbox}[colback=gray!30, colframe=Veronablue, arc=0pt, outer arc=0pt, title = \textbf{Inputs}]
    \begin{itemize}
	\item Number of contestants, $n$. Has to be odd
	\item Matrix of points per day per player, $p_{n \times n}$
    \end{itemize}
    \end{tcolorbox}

    \begin{tcolorbox}[colback=gray!30, colframe=Veronablue, arc=0pt, outer arc=0pt, title = \textbf{Outputs}]
    \begin{itemize}
	\item Schedule with the set of pairings \{\{$r_1$, $p_i$, $p_j$\}, \dots, \{$r_n$, $p_k$, $p_h$\}\} that maximizes total score. Ensured to be optimal if it's obtained through the ILP.
    \end{itemize}
    \end{tcolorbox}
    
\end{frame}

\subsection{Definitions}
\begin{frame}{\secname: \subsecname}
    In order to specify the constraints, we need to specify the sets and variables we're going to work with:
    \vspace{1cm}

    \begin{minipage}{0.49\textwidth}
	\begin{itemize}
	    \item $M(x, y)$ matches played among $x$ and $y$ (\textcolor{Veronablue}{1})
	    \item $R(r)$  matches played at round $r$ (\textcolor{Veronablue}{2})
	    \item $W(p)$ matches played by player $p$ as white (\textcolor{Veronablue}{3})
	    \item $B(p)$ matches played by player $p$ as black (\textcolor{Veronablue}{3})
	    \item $G(p,r)$ games played by $p$ at round $r$ (\textcolor{Veronablue}{4})
	    \item $F(r)$ free players at round $r$ (\textcolor{Veronablue}{5})
	\end{itemize}
    \end{minipage}
    \hfill
    \begin{minipage}{0.49\textwidth}
	\begin{enumerate}
	    \item Each contestant will play exactly once against each of the other contestants.
	    \item Each round will consist of $\frac{n-1}{2}$ matches.
	    \item Players will play 50\% of their games as white, 50\% will be played as black.
	    \item Players can play up to 1 match per round
	    \item Objective function
	\end{enumerate}
    \end{minipage}
    
\end{frame}

\section{Integer Linear Programming Model}
\subsection{Variables}
\begin{frame}{\secname: \subsecname}
    Every set will be constructed from a boolean multidimensional array. \textit{matches}$[w][b][r]$ will be 1 whenever player $w$ plays player $b$ in round $r$, and 0 otherwise.
    \begin{tcolorbox}[colback=gray!30, colframe=Veronablue, arc=0pt, outer arc=0pt, title = \textbf{Set constructions}]
	\begin{align*}
	    M(x, y)   &= \{ \{x, y, r\} &|& \ \text{matches}[x][y][r] = 1 \lor \text{matches}[y][x][r] = 1   &\forall &r \in [1, Rounds]\}\\
	    F(r)      &= \{ p           &|& \ \text{matches}[p][o][r] = 0 \land \text{matches}[o][p][r] = 0  &\forall &o \in [1, n]\}\\
	    W(p)      &= \{ \{p, b, r\} &|& \ \text{matches}[p][b][r] = 1                                    &\forall &r \in [1, Rounds], b \in [1, n]\}\\
	    B(p)      &= \{ \{w, p, r\} &|& \ \text{matches}[w][p][r] = 1                                    &\forall &r \in [1, Rounds], w \in [1, n]\}\\
	    R(r)      &= \{ \{w, b, r\} &|& \ \text{matches}[w][b][r] = 1                                    &\forall &w, b \in [1, n]\}\\
	    G(p, r)   &= \{ \{o, p, r\}           &|& \ \text{matches}[p][o][r] = 1  \lor \text{matches}[o][p][r] = 1  &\forall &o \in [1, n]\}\\
	\end{align*}
    \end{tcolorbox}
\end{frame}

\subsection{Constraints}
\begin{frame}{\secname: \subsecname}
    \begin{minipage}{0.49\textwidth}
	\begin{equation}
	    \label{playwitheachother}
	    |M(x,y)| = 1 \quad \forall x,y \in P \ | \  x \neq y
	\end{equation}

	\begin{equation}
	    \label{matchesperround}
	    |R(r)| = \frac{n-1}{2}  \quad \forall r \in [1,\ \text{Rounds}] 
	\end{equation}

	\begin{equation}
	    \label{fairness}
	    |W(p)| = \frac{n-1}{2} \quad \forall r \in [1, \ \text{Rounds}], \forall p \in P
	\end{equation}

	\begin{equation}
	    \label{1gameperround}
	    |G(p,r)| \leq 1 \quad \forall p \in P, r \in [1, \ \text{Rounds}] 
	\end{equation}


    \end{minipage}
    \hfill
    \begin{minipage}{0.49\textwidth}
	\begin{enumerate}
	    \item Each contestant will play exactly once against each of the other contestants.
	    \item Each round will consist of $\frac{n-1}{2}$ matches.
	    \item Players will play 50\% of their games as white, 50\% will be played as black.
	    \item Players can play up to 1 match per round
	\end{enumerate}
    \end{minipage}

\end{frame}

\subsection{Redundant constraints}
\begin{frame}{\secname: \subsecname}
    \begin{minipage}{0.40\textwidth}
	Redundant constraints might appear to make the model faster but they seem make it slower in the long run
	\begin{equation*}
	    \label{noselfplay}
	    |M(x,x)| = 0 \quad \forall x \in P 
	\end{equation*}

	\begin{equation*}
	    \label{blackfairness}
	    |B(p)| = \frac{n-1}{2} \quad \forall r \in [1, \ \text{Rounds}], p \in P
	\end{equation*}

    \end{minipage}
    \hfill
    \begin{minipage}{0.57\textwidth}
	\begin{figure}[h]
	    \includegraphics[width=\linewidth]{../plots/time_per_instance.pdf}
	\end{figure}
    \end{minipage}

\end{frame}

\section{Metaheuristics}
\subsection{Greedy algorithm}
\begin{frame}{\secname: \subsecname}
\begin{tcolorbox}[colback=gray!30, colframe=Veronablue, arc=0pt, outer arc=0pt, title = \textbf{Greedy cost function}]
    \begin{equation*}
      q(c,day)= c.points\_per\_day[day]
    \end{equation*}
\end{tcolorbox}


\begin{algorithm}[H]
	\caption{Greedy algorithm} 
	\begin{algorithmic}[1]
	  \State Players $\leftarrow$ Set of Players
	  \State rests $\leftarrow$ \{\}
	    \For {day in 0..days}
	      \State playersToRest $\leftarrow$ filter Players(p) $|$ p.hasNotRested
	    \State sortedPlayers $\leftarrow$ sort playersToRest(p) by q(p,day) (DESC) 
	      \State select p $\in$ sortedPlayers[0]
	      \State rests[day]\ $\leftarrow$\ p
	    \EndFor
	\end{algorithmic} 
\end{algorithm}
\end{frame}

\subsection{Local Search}
\begin{frame}{\secname: \subsecname}
\begin{algorithm}[H]
	\caption{Local Search} 
	\begin{algorithmic}[1]
    \For {i in 0..days}
    \State best\_swap\_points $\leftarrow$ 0
    \State best\_swap $\leftarrow$ i
      \For {j in 0..days}
        \State change = EvaluateRestSwap(i,j)
        \If{change $>$ best\_swap\_points}
          \State best\_swap\_points $\leftarrow$ change
          \State best\_swap $\leftarrow$ j
        \EndIf
      \EndFor
      \State rests[i] $\leftrightarrow$ rests[best\_swap]
		\EndFor
	\end{algorithmic} 
\end{algorithm}



\end{frame}

\subsection{GRASP}
\begin{frame}{\secname: \subsecname}
\begin{algorithm}[H]
    \caption{constructRCL(day)} 
    \label{rcl}
    \begin{algorithmic}[1]
	\State $q_{max} \leftarrow $ sortedPlayers.first().points[d]
	\State $q_{min} \leftarrow$ sortedPlayers.last().points[d]
	\State $RCL_{max}$ $\leftarrow$ $\{p \in sortedPlayers\ |\ p.points[d] >= q_{max} - \alpha \cdot (q_{max} - q_{min})\}$
    \end{algorithmic} 
\end{algorithm}




\begin{algorithm}[H]
	\caption{GRASP} 
	\begin{algorithmic}[1]
	  \State rests $\leftarrow$ \{\}
	    \For {day in 0..days}
	      \State RCL $\leftarrow$ constructRCL(day)
	      \State select p $\in$ RCL randomly
	      \State rests[day]\ $\leftarrow$\ p
	    \EndFor
	\end{algorithmic} 
\end{algorithm}
\end{frame}

\subsubsection{Parameter tuning}
\begin{frame}{\subsecname: \subsubsecname}

    \begin{minipage}{0.44\textwidth}
	\begin{itemize}
	    \item Hemos probao no se que no se cuantos
	    \item Muchos alphas nos valen porque hay muchos 0 calvo de mierda
	\end{itemize}
    \end{minipage}
    \hfill
    \begin{minipage}{0.52\textwidth}
	\centering
	\begin{figure}[H]
	    \centering
	    \includegraphics[width=\linewidth]{../plots/error.pdf}
	    \label{fig:error}
	\end{figure}
    \end{minipage}
\end{frame}

\section{Results}
\subsection{Time}
\begin{frame}{\secname: \subsecname}
    \begin{minipage}{0.44\textwidth}
	\begin{itemize}
	    \item ILP rest blabla
	    \item texto
	\end{itemize}
    \end{minipage}
    \hfill
    \begin{minipage}{0.52\textwidth}
	\centering
	\begin{figure}[H]
	    \centering
	    \includegraphics[width=\linewidth]{../plots/times.pdf}
	\end{figure}
    \end{minipage}
\end{frame}

\subsection{Quality of solutions}
\begin{frame}{\secname: \subsecname}
    \begin{minipage}{0.44\textwidth}
	\begin{itemize}
	    \item ILP rest blabla
	    \item texto
	\end{itemize}
    \end{minipage}
    \hfill
    \begin{minipage}{0.52\textwidth}
	\centering
	\begin{figure}[H]
	    \centering
	    \includegraphics[width=\linewidth]{../plots/solutions.pdf}
	\end{figure}
    \end{minipage}
\end{frame}



%\subsection{Ruidos a filtrar}
%\begin{frame}{\secname: \subsecname}
%    %\begin{figure}[H]
%	%\begin{subfigure}{0.3\textwidth}
%	    %\includegraphics[width=\linewidth]{imagenes/ruidogaussiano_1.jpg}
%	    %\caption{Original}
%	%\end{subfigure}
%	%\hfill
%	%\begin{subfigure}{0.3\textwidth}
%	    %\includegraphics[width=\linewidth]{imagenes/ruidogaussiano_2.jpg}
%	    %\caption{Ruido gaussiano}
%	%\end{subfigure}
%	%\hfill
%	%\begin{subfigure}{0.3\textwidth}
%	    %\includegraphics[width=\linewidth]{imagenes/ruidoimpulsivo_1.jpg}
%	    %\caption{Ruido impulsivo}
%	%\end{subfigure}
%	%\caption{Ruidos a filtrar y sus efectos visuales}
%    %\end{figure}
%\end{frame}
%
%\section{Marco Teórico}
%\subsection{Algoritmo}
%\begin{frame}{\secname: \subsecname}
%    \begin{minipage}{0.45\textwidth}
%	Primero, un poco de notación:
%
%	\begin{itemize}
%	    \item $x_i$ se refiere al pixel que está siendo filtrado
%	    \item $x^j$ se refiere a cualquier pixel dentro de la ventana
%		de filtrado
%	\end{itemize}
%
%	Trabajaremos con los $q$ vecinos más similares a $x_i$
%    \end{minipage}\hfill
%    \begin{minipage}{0.45\textwidth}
%	%\begin{figure}[H]
%	    %\centering
%	    %\includegraphics[width=\linewidth]{../archivos/tests/window.pdf}
%	    %\caption{Ventana de filtrado. Ejemplo de dimensiones 3x3}
%	%\end{figure}
%    \end{minipage}
%\end{frame}
%
%\begin{frame}{\secname: \subsecname}
%    \center{
%	Caracterización de los ruidos gaussianos e impulsivos.
%    }
%
%    \begin{minipage}{0.45\textwidth}
%	%\begin{figure}[H]
%	    %\centering
%	    %\includegraphics[width=0.7\linewidth]{../archivos/ventana.pdf}
%	%\end{figure}
%	Grado de impulsividad determinado por la métrica $\displaystyle \text{ROAD}_m = \sum_{j=1}^{m}d(x_i, x^j) $
%    \end{minipage}\hfill
%    \begin{minipage}{0.45\textwidth}
%	%\begin{figure}[H]
%	    %\centering
%	    %\includegraphics[width=\linewidth]{../archivos/similaritydegree.pdf}
%	%\end{figure}
%	Tres grados de semejanza: alta, media y baja. Dependiente de
%	la distancia entre píxeles $d(x_i, x^j)$.
%    \end{minipage}
%\end{frame}
%
%
%\begin{frame}{\secname: \subsecname}
%    \begin{minipage}{0.4\textwidth}
%	\begin{itemize}
%	    \item Reglas difusas 
%
%		\begin{enumerate}
%		    \item $\dots$
%		    \item \textbf{SI} ($x^j$ no es impulsivo \textbf{Y} $x_i$ es impulsivo \textbf{Y} 
%			la semejanza entre $x^j$ y $x_i$ es moderada) \textbf{ENTONCES} $\omega_j$ 
%			es un peso \textbf{moderado}.
%		    \item $\dots$
%		\end{enumerate}
%	    \item Defuzzificación mediante el centro de gravedad
%	\end{itemize}
%
%    \end{minipage}\hfill
%    \begin{minipage}{0.6\textwidth}
%	\begin{center}
%	\scalebox{0.55}{
%	    \begin{algorithm}[H] %or another one check
%	     \caption{Filtro difuso secuencial}
%		\KwData{Imagen ruidosa $I$, parámetros $n, q, m, p_1, p_2, p_3, p_4$}
%		\KwResult{Imagen filtrada $I^{\prime}$}
%
%		Imagen $I_0 = I$\\
%		\For{Iteración $\  it = 1,\ldots$}{
%			Imagen $I_{it} = I_{it-1}$\\
%			\For{$x_i$ pixel $\in$ ${I_{it}}$}{
%			    Tomar la ventana $W$ $n \times n$ centrada en $x_i$\\
%			    %Tomar la $n \times n$ filtering window $W$ with central pixel $x_i$\\
%			    \textbf{\underline{Cálculo del grado de impulsividad}}\\
%				\Indp
%				Calcular $\mu (x_i)$\\
%				\Indm
%			    \textbf{\underline{Cálculo del grado de semejanza}}\\
%				\Indp
%				Ordenar los píxeles $  x^j \in W$ según $d(x_i, x^j)$\\
%				Seleccionar los $q$ píxeles mas cercanos $x^1,\ldots,x^q$\\
%				\For {$ j = 1,\ldots, q$}{
%				    Calcular $\mu_H (x_i, x^j), \mu_L (x_i, x^j), \mu_H (x_i , x^j),$
%				}
%				\Indm
%			    %\textbf{\underline{Computation of averaging weights by defuzzyficattion}}}
%			    \textbf{\underline{Cálculo de los pesos mediante defuzzificación}}\\
%				\Indp
%				\For{$ j = 1,\ldots, q$}{
%				    %Compute certainty degree of the antecedents of fuzzy rules for  $x^j$\\
%				    Calcular las reglas difusas para \{$x_i, x^j$\}\\
%				    Calcular el peso $w_j$ correspondiente a $x^j$ mediante COG\\
%				}
%				\Indm
%			    \textbf{\underline{Cálculo del nuevo valor para $x_i$}}\\
%				\Indp
%				$\displaystyle \hat{x}_i = \frac{\sum_{j=1}^{q} \omega_j \cdot x^{j}}{\sum_{j=1}^{q} \omega_j}$\\
%				\Indm
%			}
%		}
%		\label{algoritmo_filtro_secuencial}
%	    \end{algorithm}
%	}
%	\end{center}
%    \end{minipage}
%\end{frame}
%
%%\begin{frame}{\secname: \subsecname}
%    %\begin{center}
%    %\scalebox{0.6}{
%	%\begin{algorithm}[H] %or another one check
%	 %\caption{Filtro difuso secuencial}
%	    %\KwData{Imagen ruidosa $I$, parámetros $n, q, m, p_1, p_2, p_3, p_4$}
%	    %\KwResult{Imagen filtrada $I^{\prime}$}
%
%	    %Imagen $I_0 = I$\\
%	    %\For{Iteración $\  it = 1,\ldots$}{
%		    %Imagen $I_{it} = I_{it-1}$\\
%		    %\For{$x_i$ pixel $\in$ ${I_{it}}$}{
%			%Tomar la ventana $W$ $n \times n$ centrada en $x_i$\\
%			%%Tomar la $n \times n$ filtering window $W$ with central pixel $x_i$\\
%			%\textbf{\underline{Cálculo del grado de impulsividad}}\\
%			    %\Indp
%			    %Calcular $\mu (x_i)$\\
%			    %\Indm
%			%\textbf{\underline{Cálculo del grado de semejanza}}\\
%			    %\Indp
%			    %Ordenar los píxeles $  x^j \in W$ según $d(x_i, xj)$\\
%			    %Seleccionar los $q$ píxeles mas cercanos $x^1,\ldots,x^q$\\
%			    %\For {$ j = 1,\ldots, q$}{
%				%Calcular $\mu_H (x_i, x^j), \mu_L (x_i, x^j), \mu_H (x_i , x^j),$
%			    %}
%			    %\Indm
%			%%\textbf{\underline{Computation of averaging weights by defuzzyficattion}}}
%			%\textbf{\underline{Cálculo de los pesos mediante defuzzificación}}\\
%			    %\Indp
%			    %\For{$ j = 1,\ldots, q$}{
%				%%Compute certainty degree of the antecedents of fuzzy rules for  $x^j$\\
%				%Calcular las reglas difusas para \{$x_i, x^j$\}\\
%				%Calcular el peso $w_j$ correspondiente a $x^j$ mediante COG\\
%			    %}
%			    %\Indm
%			%\textbf{\underline{Cálculo del nuevo valor para $x_i$}}\\
%			    %\Indp
%			    %$\displaystyle \hat{x}_i = \frac{\sum_{j=1}^{q} \omega_j \cdot x^{j}}{\sum_{j=1}^{q} \omega_j}$\\
%			    %\Indm
%		    %}
%	    %}
%	    %\label{algoritmo_filtro_secuencial}
%	%\end{algorithm}
%    %}
%    %\end{center}
%
%%\end{frame}
%
%
%\section{Implementación}
%    %\begin{frame}{\secname : \subsecname}
%	%Consideraciones generales
%
%	%Secuencial
%
%	%Paralela
%
%	%Distribuida
%    %\end{frame}
%%\subsection{Consideraciones generales}
%%\begin{frame}{\secname : \subsecname}
%    %Evitar código spaguetti, clases representando ventanas, imágenes sobre un array plano, todo código desde 0 para tener
%    %más control...
%%\end{frame}
%
%\subsection{Versión secuencial}
%\begin{frame}{\secname : \subsecname}
%    \center{
%	Cálculo del centro de gravedad mediante métodos geométricos.
%    }
%    \begin{minipage}{0.505\textwidth}
%	%\begin{figure}[H]
%	    %\includegraphics[width=\textwidth]{../archivos/cogcomputation}
%	    %\caption{Centro de gravedad a obtener por cada $x_i, x_j \in W$}
%	%\end{figure}
%    \end{minipage}
%    \hfill
%    \begin{minipage}{0.46\textwidth}
%	%\begin{figure}[H]
%	    %\includegraphics[width=\textwidth]{../archivos/benchmarks/calculateweight.pdf}
%	    %\caption{Aceleración obtenida gracias al cálculo geométrico}
%	%\end{figure}
%    \end{minipage}
%\end{frame}
%
%\begin{frame}{\secname: \subsecname}
%    %\begin{minipage}{0.60\textwidth}
%	%\begin{figure}[H]
%	    %\includegraphics[width=\textwidth]{../archivos/pgmwindowcreation.pdf}
%	    %\caption{}
%	%\end{figure}
%    %\end{minipage}
%    %\hfill
%    %\begin{minipage}{0.274\textwidth}
%	%\begin{figure}[H]
%	    %\includegraphics[width=\textwidth]{../archivos/pgmwindow.pdf}
%	    %\caption{Gestión de las ventanas de manera transparente al\\ desarrollador}
%	%\end{figure}
%    %\end{minipage}
%
%    \begin{figure}[H]
%	\hspace{0.05\textwidth}
%	%\begin{subfigure}{0.3\textwidth}
%	    %\includegraphics[width=\textwidth]{../archivos/pgmwindow.pdf}
%	    %\caption{Ventanas que sobrepasan los límites de la imagen a filtrar}
%	%\end{subfigure}
%	\hfill
%	%\begin{subfigure}{0.4\textwidth}
%	    %\includegraphics[width=\textwidth]{../archivos/pgmwindowcreation.pdf}
%	    %\caption{Ajuste automático. $W \rightarrow W'$}
%	%\end{subfigure}
%	\hspace{0.05\textwidth}
%	\caption{Gestión de las ventanas de filtrado transparente al desarrollador}
%    \end{figure}
%
%\end{frame}
%
%\begin{frame}{\secname: \subsecname}
%    \begin{minipage}{0.44\textwidth}
%	La implementación inicial ingenua podía llegar a calcular el grado
%	de impulsividad de cada pixel en numerosas ocasiones. Por ello introdujimos
%	un <<almacén>> de grados de impulsividad para evitar este problema. 
%    \end{minipage}
%    \hfill
%    \begin{minipage}{0.50\textwidth}
%	%\begin{figure}[H]
%	    %\includegraphics[width=\textwidth, left]{../archivos/secoptimizada.pdf}
%	%\end{figure}
%    \end{minipage}
%\end{frame}
%
%\subsection{Versión en memoria compartida}
%\begin{frame}{\secname : \subsecname}
%    \begin{minipage}{0.50\textwidth}
%	Paralelización del núcleo de la aplicación mediante directivas 
%	de \textcolor[HTML]{00737D}{OpenMP}. 
%	\begin{itemize}
%	    \item Paralelización del tratamiento de la imagen. 
%	    \item Almacenamiento local para cada hilo (TLS)
%	    \item Sincronización sin cerrojos
%	\end{itemize}
%    \end{minipage}\hfill
%    \begin{minipage}{0.47\textwidth}
%	%\begin{figure}[H]
%	    %\includegraphics[width=\textwidth, left]{../archivos/figure4_old.png}
%	    %\caption{La paralelización actúa a nivel de subdominio $\Omega_i$}
%	%\end{figure}
%    \end{minipage}
%\end{frame}
%
%\subsection{Versión distribuida}
%\begin{frame}{\secname : \subsecname}
%    \begin{minipage}{0.44\textwidth}
%	Desarrollada utilizando MPI.
%
%	\vspace{0.5cm}
%
%	Tamaño de ventana $n = 2\omega + 1$
%
%
%	\begin{itemize}
%	    \item $\Omega_i$: Región a \textbf{filtrar} por el nodo $i$
%	    \item $\Omega_i^\omega$: Región a \textbf{necesaria} para el nodo $i$
%	    \item \testclr{black!25!}{Solapamiento} $\rightarrow$ \textbf{Comunicación}
%	\end{itemize}
%
%	\begin{tcolorbox}[colback=gray!30, colframe=Veronablue, title = \textbf{Propuesta}]
%	    Esquema de iteraciones locales. En lugar de comunicar cada iteración, hacerlo
%	    cada $x$ iteraciones.
%	\end{tcolorbox}
%    \end{minipage}\hfill
%    \begin{minipage}{0.55\textwidth}
%
%	%\noindent\begin{figure}[H]
%	    %\includegraphics[width=\textwidth, left]{../archivos/domaindecomposition.pdf}
%	    %\caption{Descomposición del trabajo para 3 nodos}
%	    %\label{figure5}
%	%\end{figure}
%    \end{minipage}
%\end{frame}
%
%\section{Rendimiento}
%
%\subsection{Equipos de pruebas}
%\begin{frame}{\secname : \subsecname}
%    %\begin{minipage}[t]{0.49\textwidth}
%
%	%\begin{tcolorbox}[colback=gray!40, colframe=Veronablue!90, title = \faDesktop \ Equipo 1. Ordenador personal]
%	    %\begin{itemize}
%		%\itemsep0.2em
%		%\setlength{\itemindent}{-1em}
%		%\item (x1) AMD Ryzen 2700X (8C / 16T), 16GB RAM DDR4 3200MT/s
%		%\item ArchLinux. Kernel 5.16.12 
%		%\item GCC 11.2.0
%	    %\end{itemize}
%	%\end{tcolorbox}
%	%\vfill
%    %\end{minipage}\hfill
%    %\begin{minipage}[t]{0.49\textwidth}
%	%%\vspace{1.435cm}
%	%\begin{tcolorbox}[colback=gray!40, colframe=Veronablue!90, title = \faDesktop \  Equipo 2. Cluster del IUII]
%	    %%\vspace{.95cm}
%	    %\begin{itemize}
%		%\itemsep0.2em
%		%\setlength{\itemindent}{-1em}
%		%\item (2x) CPU Intel Xeon X 5660 (6C/12T), 48GB RAM DDR3 1333MT/s
%		%\item CentOS 7. Kernel 3.10.0
%		%\item GCC 7.5.0
%		%\item OpenMPI 4.0.2
%	    %\end{itemize}
%	%\end{tcolorbox}
%	%\vfill
%    %\end{minipage}
%
%    \begin{multicols}{2}
%	\begin{tcolorbox}[colback=gray!30, colframe=Veronablue, title = \faDesktop \ Equipo 1. Ordenador personal,
%	   height=4cm ]
%	    \begin{itemize}
%		\itemsep0.2em
%		\setlength{\itemindent}{-1em}
%		\item (x1) AMD Ryzen 2700X (8C / 16T), 16GB RAM DDR4 3200MT/s
%		\item ArchLinux. Kernel 5.16.12 
%		\item GCC 11.2.0
%	    \end{itemize}
%	\end{tcolorbox}
%	\columnbreak
%
%	\begin{tcolorbox}[colback=gray!30, colframe=Veronablue, title = \faDesktop \  Equipo 2. Cluster del IUII, height=4cm]
%	    %\vspace{.95cm}
%	    \begin{itemize}
%		\itemsep0.2em
%		\setlength{\itemindent}{-1em}
%		\item (2x) CPU Intel Xeon X 5660 (6C/12T), 48GB RAM DDR3 1333MT/s
%		\item CentOS 7. Kernel 3.10.0
%		\item GCC 7.5.0
%		\item OpenMPI 4.0.2
%	    \end{itemize}
%	\end{tcolorbox}
%
%    \end{multicols}
%\end{frame}
%
%\subsection{Versión en memoria compartida}
%    \begin{frame}{\secname : \subsecname}
%	\begin{minipage}{0.25\textwidth}
%	    Rendimiento {\color{ao}óptimo} en el Equipo 1.
%
%	    \vspace{0.5cm}
%
%	    Rendimiento {\color{alizarin}subóptimo} en el Equipo 2.
%	    \begin{itemize}
%		\item Nodos dual-socket
%		\item Compilador más anticuado
%	    \end{itemize}
%
%	\end{minipage}
%	\hfill
%	\begin{minipage}{0.70\textwidth}
%	    %\begin{figure}[H]
%		%\includegraphics[width=\textwidth]{../archivos/benchmarks/openmp_impl_bar.pdf}
%		%\caption{Benchmark medido en el Equipo 1}
%	    %\end{figure}
%	\end{minipage}
%    \end{frame}
%\subsection{Versión distribuida}
%    \begin{frame}{\secname : \subsecname}
%
%	\begin{minipage}{0.49\textwidth}
%	    %\begin{figure}[H]
%		%\includegraphics[width=\textwidth]{../archivos/tests/mpi/escalado_nodosmpi.pdf} 
%	    %\end{figure}
%	\end{minipage}\hfill
%	\begin{minipage}{0.49\textwidth}
%	    %\begin{figure}[H]
%		%\includegraphics[width=\textwidth]{../archivos/tests/mpi/escalado_general.pdf} 
%	    %\end{figure}
%	\end{minipage}
%    \end{frame}
%\subsubsection{Esquema de iteraciones locales}
%    \begin{frame}{\secname : \subsecname}
%
%	\begin{minipage}{0.45\textwidth}
%	\begin{itemize}
%	    \item La eficiencia combinada se ve lastrada por el rendimiento paralelo en los
%		  equipos del clúster. 
%	    %\item Bastante variabilidad de resultados a la hora de medir en el clúster.
%	\end{itemize}
%
%	\vspace{1cm}
%
%	\begin{itemize}
%	    \item El esquema de iteraciones locales supone una aceleración cercana al 
%		\textbf{12\%} sin pérdida de calidad de filtrado.
%	\end{itemize}
%	\end{minipage}\hfill
%	\begin{minipage}{0.50\textwidth}
%	    %\begin{figure}[H]
%		%\includegraphics[width=\textwidth]{../archivos/tests/mpi/iteracionesinternas.pdf} 
%	    %\end{figure}
%	\end{minipage}
%    \end{frame}
%
%\section{Calidad de filtrado}
%\subsection{Métricas numéricas}
%    \begin{frame}{\secname : \subsecname \ I}
%	\begin{table}[H] 
%	\centering
%	\begin{tabular}{c@{\quad}cc@{\quad}cc@{\quad}cc}
%	\toprule
%	\multicolumn{1}{c}{\rule{0pt}{12pt} }&\multicolumn{6}{c}{MAE}\\[2pt]
%	\multicolumn{1}{c}{\rule{0pt}{12pt} }&\multicolumn{2}{c}{Vista axial}&\multicolumn{2}{c}{Vista sagital}&\multicolumn{2}{c}{Vista coronal}\\[2pt]
%	\multicolumn{1}{c}{\rule{0pt}{12pt}Ruido}& Ruidosa & Filtrada & Ruidosa & Filtrada & Ruidosa & Filtrada\\[2pt]                   
%	\midrule
%	$\sigma=5, \ p=0.05$ & 13.19  &  4.32 &   8.54      &  2.60     &  9.20      & 2.98     \\
%	$\sigma=10, \ p=0.1$ & 21.29  &  5.97 &  14.28      &  3.61     &  15.37     & 4.07     \\
%	$\sigma=20, \ p=0.2$ & 35.65  &  8.26 &  24.87      &  5.91     &  26.64     & 6.47     \\
%	$\sigma=30, \ p=0.3$ & 50.03  & 13.02 &  34.84      &  9.08     &  37.29     & 9.81     \\
%	\bottomrule
%	\end{tabular}
%	\caption{Valores mínimos de MAE en las imágenes contaminadas con los distintos ruidos gaussiano e impulsivos}
%	\end{table}
%    \end{frame}
%
%    \begin{frame}{\secname : \subsecname \ II}
%	\begin{table}[H] 
%	\centering
%	\begin{tabular}{c@{\quad}cc@{\quad}cc@{\quad}cc}
%	\toprule
%	\multicolumn{1}{c}{\rule{0pt}{12pt} }&\multicolumn{6}{c}{PSNR}\\[2pt]
%	\multicolumn{1}{c}{\rule{0pt}{12pt} }&\multicolumn{2}{c}{Vista axial}&\multicolumn{2}{c}{Vista sagital}&\multicolumn{2}{c}{Vista coronal}\\[2pt]
%	\multicolumn{1}{c}{\rule{0pt}{12pt}Ruido}& Ruidosa & Filtrada & Ruidosa & Filtrada & Ruidosa & Filtrada\\[2pt]                   
%	\midrule
%	$\sigma=5, \ p=0.05$ & 19.03  & 34.57 &  18.85  & 34.81 & 19.21  & 34.32 \\
%	$\sigma=10, \ p=0.1$ & 16.06  & 32.22 &  15.86  & 32.53 & 16.13  & 32.03 \\
%	$\sigma=20, \ p=0.2$ & 12.98  & 29.53 &  12.82  & 28.80 & 13.07  & 28.46 \\
%	$\sigma=30, \ p=0.3$ & 11.17  & 25.80 &  11.02  & 25.24 & 11.25  & 25.06 \\
%	\bottomrule
%	\end{tabular}
%	\caption{Valores máximos de PSNR en las imágenes contaminadas con los distintos ruidos gaussiano e impulsivos}
%	\label{psnr}
%	\end{table}
%    \end{frame}
%
%
%\subsection{Calidad visual}
%\begin{frame}{\secname : \subsecname}
%    %\begin{figure}[H]
%	%\begin{subfigure}{0.328\textwidth}
%	    %\includegraphics[width=\linewidth]{../archivos/imagenes/coronal_g10_i01.pgm.png}
%	    %\caption{Contaminada $\sigma=10,\rho=0.1$}
%	%\end{subfigure}
%	%\begin{subfigure}{0.328\textwidth}
%	    %\includegraphics[width=\linewidth]{../archivos/imagenes/coronal_g10_i01.pgm.filtered.pgm.png}
%	    %\caption{Filtrada}
%	%\end{subfigure}
%	%\begin{subfigure}{0.328\textwidth}
%	    %\includegraphics[width=\linewidth]{../archivos/coronal.png}
%	    %\caption{Original}
%	%\end{subfigure}
%	%\caption{Vista coronal contaminada, filtrada y original}
%    %\end{figure}
%\end{frame}
%
%%%%%%%%%%%%%%%%%%%%%%%%%%%%%%%%%%%%%%%%%%%%%%%%%%%%%%%%%%%%%%%%%%%%%%%%%%%%%%%%%%%%%
%%%%%%%%%%%%%%%%%%%%%%%%%%%%%%%%%%%%%%%%%%%%%%%%%%%%%%%%%%%%%%%%%%%%%%%%%%%%%%%%%%%%%
%
%%\begin{frame}{\secname : \subsecname}
%    %\begin{figure}[H]
%	%\includegraphics[width=0.5\linewidth]{../archivos/imagenes/coronal_g10_i01.pgm.png}
%	%\caption{\scriptsize{Contaminada $\sigma=10,\rho=0.1$}}
%    %\end{figure}
%%\end{frame}
%
%%\begin{frame}{\secname : \subsecname}
%    %\begin{figure}[H]
%	%\begin{subfigure}{0.45\textwidth}
%	    %\includegraphics[width=\linewidth]{../archivos/imagenes/coronal_g10_i01.pgm.filtered.pgm.png}
%	    %\caption{Filtrada}
%	%\end{subfigure}
%	%\hfill
%	%\begin{subfigure}{0.45\textwidth}
%	    %\includegraphics[width=\linewidth]{../archivos/coronal.png}
%	    %\caption{Original}
%	%\end{subfigure}
%	%\caption{Vista coronal contaminada, filtrada y original}
%    %\end{figure}
%%\end{frame}
%
%
%%%%%%%%%%%%%%%%%%%%%%%%%%%%%%%%%%%%%%%%%%%%%%%%%%%%%%%%%%%%%%%%%%%%%%%%%%%%%%%%%%%%%
%%%%%%%%%%%%%%%%%%%%%%%%%%%%%%%%%%%%%%%%%%%%%%%%%%%%%%%%%%%%%%%%%%%%%%%%%%%%%%%%%%%%%
%
%
%\section{Conclusión}
%\begin{frame}{\secname}
%    \textbf{Resultados:}
%    \begin{itemize}
%	\item Alta eficiencia de la implementación paralela 
%	    % Comentar que es con el número mayor de nodos.
%	    \begin{itemize}
%		\item $> 85\%$ Equipo 1
%		\item $> 96\%$ Equipo 2
%	    \end{itemize}
%	\item Alta eficiencia de la implementación distribuida. $> 82\%$
%	      con 20 nodos.
%	\item Calidad de filtrado notable.
%    \end{itemize}
%    \textbf{Líneas futuras:}
%    \begin{itemize}
%	\item Implementación con primitivas MPI no bloqueantes.
%	\item Desarrollar una versión acelerada por GPU.
%    \end{itemize}
%\end{frame}
%
%\begin{frame}{\secname}
%    \begin{minipage}{0.54\textwidth}
%	\huge{Gracias por su atención.}
%    \end{minipage}\hfill
%    \begin{minipage}{0.32\textwidth}
%	%\noindent\begin{figure}[H]
%	    %\includegraphics[width=\textwidth, left]{portada.pdf}
%	%\end{figure}
%    \end{minipage}
%
%\end{frame}
%
%
%%\section{Aclaraciones}
%%\subsection{Lógica difusa}
%\begin{frame}[noframenumbering]{Aclaraciones}
%    \begin{minipage}{0.55\textwidth}
%    \begin{itemize}
%	\item Valores lógicos $\{0, 1\} \rightarrow [0, 1]$
%	\item Pertenencia a conjuntos definida por funciones 
%	\item Operadores lógicos  
%		%\item $\{A \land B, A \lor B\} \rightarrow \{A \cdot B, A + B - A \cdot B\}$ 
%	    \begin{itemize}
%		\item $\{A \land B\} \rightarrow \{A \cdot B\}$ 
%		\item $\{A \lor B\} \rightarrow \{A + B - A \cdot B\}$ 
%	    \end{itemize}
%	\item Reglas difusas
%	\item Defuzzificación
%    \end{itemize}
%    \end{minipage}\hfill
%    \begin{minipage}{0.4\textwidth}
%	%\begin{figure}[H]
%	    %\centering
%	    %\includegraphics[width=\textwidth]{../archivos/similaritydegree}
%	    %\caption{Grado de semejanza para un pixel $x^j$ en función de $d(x_i, x^j)$}
%	%\end{figure}
%
%    \end{minipage}
%\end{frame}
%\begin{frame}[noframenumbering]{Aclaraciones II}
%    \begin{minipage}{0.55\textwidth}
%	\begin{itemize}
%	    \item Caracterización de la impulsividad y la semejanza entre píxeles dentro de la ventana de filtrado.
%	    \item Reglas difusas 
%
%		\begin{enumerate}
%		    \item $\dots$
%		    \item \textbf{SI} ($x^j$ no es impulsivo \textbf{Y} $x_i$ es impulsivo \textbf{Y} 
%			la semejanza entre $x^j$ y $x_i$ es moderada) \textbf{ENTONCES} $\omega_j$ 
%			es un peso \textbf{moderado}.
%		    \item $\dots$
%		\end{enumerate}
%	    \item Defuzzificación mediante el centro de gravedad
%	\end{itemize}
%
%    \end{minipage}\hfill
%    \begin{minipage}{0.4\textwidth}
%	%\begin{figure}[H]
%	    %\includegraphics[width=\textwidth]{../archivos/membershipfunctions}
%	    %\caption{Funciones de membresía $\eta_H (\omega_j),\ \eta_M (\omega_j),$ y $\eta_L
%		    %(\omega_j)$ junto a $K_{L}, K_{M}, K_{H}$ correspondientes a las 
%		    %reglas difusas \{1, 2, 3\} respectivamente para determinados píxeles $x_{i}$ y $x^j$}
%	%\end{figure}
%
%    \end{minipage}
%\end{frame}

%\begin{frame}[noframenumbering]{\secname}

%\end{frame}


\end{document}
